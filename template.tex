%!TEX TS-program = xelatex
%!TEX encoding = UTF-8 Unicode

\documentclass[$fontsize$, a4paper]{article}

% LAYOUT
%--------------------------------
\usepackage{geometry} 
\geometry{$geometry$}

% footer
\usepackage{fancyhdr}
\pagestyle{fancy}
\fancyhf{}
\renewcommand{\headrulewidth}{0pt}

% No page numbers
\pagenumbering{gobble}

% Left align
\usepackage[document]{ragged2e}

$if(letterhead)$
  % To include the letterhead
  \usepackage{wallpaper}
  \ULCornerWallPaper{1}{letterhead.pdf}
$endif$

% TYPOGRAPHY
%--------------------------------
\usepackage{fontspec} 
\usepackage{xunicode}
\usepackage{xltxtra}

% converts LaTeX specials (quotes, dashes etc.) to Unicode
\defaultfontfeatures{Mapping=tex-text}
\setromanfont [Ligatures={Common}, Numbers={OldStyle}]{$seriffont$}
\setsansfont[Scale=0.9]{$sansfont$}

% Set paragraph break
\setlength{\parskip}{1em}

% Custom ampersand
\newcommand{\amper}{{\fontspec[Scale=.95]{$seriffont$}\selectfont\itshape\&}}

$if(seriffont)$
\setmainfont[SmallCapsFeatures={LetterSpace=5,Letters=SmallCaps}]{$seriffont$}
$endif$
$if(sansfont)$
    \setsansfont{$sansfont$}
$endif$

% Command required by how Pandoc handles the list conversion
\providecommand{\tightlist}{%
  \setlength{\itemsep}{0pt}\setlength{\parskip}{0pt}}

% TABLE CUSTOMIZATION
%--------------------------------
\usepackage{spreadtab}
\usepackage[compact]{titlesec} % For customizing title sections
\titlespacing*{\section}{0pt}{3pt}{-7pt} % Remove margin bottom from the title
\usepackage{arydshln} % For the dotted line on the table
\renewcommand{\arraystretch}{1.5} % Apply vertical padding to table cells
\usepackage{hhline} % For single-cell borders
\usepackage{enumitem} % For customizing lists
\setlist{nolistsep} % No whitespace around list items
\setlist[itemize]{leftmargin=0.5cm} % Reduce list left indent
\setlength{\tabcolsep}{9pt} % Larger gutter between columns


% LANGUAGE
%--------------------------------
$if(lang)$
\usepackage{polyglossia}
\setmainlanguage{$polyglossia-lang.name$}
$endif$

% PDF SETUP
%--------------------------------
\usepackage[xetex, bookmarks, colorlinks, breaklinks]{hyperref}
\hypersetup
{
  pdfauthor={$author$},
  pdfsubject=Rechnung Nr. $invoice-nr$,
  pdftitle=Rechnung Nr. $invoice-nr$,
  linkcolor=blue,
  citecolor=blue,
  filecolor=black,
  urlcolor=blue
}

% To display custom date
% \usepackage[nodayofweek]{datetime}
% \newdate{date}{01}{12}{1867}
% \date{\displaydate{date}}
% Use this instead of \today: % \displaydate{date}

% DOCUMENT
%--------------------------------
\begin{document}
\small
\textsc{\textbf{$author$}} \\
$for(from)$
\textsc{$from$}
$sep$
\textbullet{} 
$endfor$

\vspace{1em}

\normalsize \sffamily
$for(to)$
$to$\\
$endfor$

\vspace{6em}

\begin{flushright}
  \small
  $city$, \today
\end{flushright}

\section*{\textsc{Rechnung} \textsc{\#$invoice-nr$}}

\vspace{18mm}

\small
$intronote$

\medskip

\footnotesize
\newcounter{pos}
\setcounter{pos}{0}
\STautoround*{2} % Get spreadtab to always display the decimal part
$if(commasep)$\STsetdecimalsep{,}$endif$ % Use comma as decimal separator

\begin{spreadtab}{{tabular}{lp{4.2cm}rrr}}
  \hdashline[1pt/1pt]
  @ \noalign{\vskip 2mm} \textbf{Pos.} & @ \textbf{Aktivität} & @ \textbf{$currency$/Einheit} & @ \textbf{Anzahl} & @ \textbf{Preis in $currency$} \\ \hline
      $for(service)$ @ \noalign{\vskip 2mm} \refstepcounter{pos} \thepos 
        & @ $service.description$ 
        $if(service.details)$\newline \begin{itemize} 
          $for(service.details)$\scriptsize \item $service.details$ 
          $endfor$ \end{itemize}
        $endif$ & $service.per_hour$ & $service.hours$ & [-1,0]*[-2,0] \\
      $endfor$ \noalign{\vskip 2mm} \hline
  $if(VAT)$
    @ & @ & @ \multicolumn{2}{r}{Summe Honorare:}                & :={sum(e1:[0,-1])} \\ \hhline{~~~~-}
    @ & @ & @ \multicolumn{2}{r}{Umsatzsteuer $VAT$\%:}               & $VAT$/100*[0,-1] \\ \hhline{~~~~-}
  $endif$
  @ & @ & @ \multicolumn{2}{r}{\textbf{Gesamtsumme:}}   & \textbf{:={$if(VAT)$[0,-1]+[0,-2]$else$sum(c1:[0,-1])$endif$}} \\
\end{spreadtab}


\vspace{15mm}

\sffamily
\small
$closingnote$

\medskip

\IfFileExists{signature.jpg}
{
      \includegraphics[height=3.5\baselineskip]{signature.jpg} \par
}

$author$

\lfoot{
\sffamily
\small
$for(tax_info)$
  $tax_info$ \\
$endfor$
}

\rfoot{
\sffamily
\small
$for(bank_info)$
  $bank_info$ \\
$endfor$
}

\end{document}
